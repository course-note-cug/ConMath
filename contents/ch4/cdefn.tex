\section{Divisibility}

\begin{frame}{The Divisible}

\begin{definition}[Divisibility]
     $m \mid n \iff m>0 \land n=mk$ for some int $k$. 
\end{definition}

\begin{itemize}
    \item That is, $n$ is a multiple of $m$, and it is not possibly 
    positive. 
\end{itemize}
    
\end{frame}

\begin{frame}{Greatest Common Divisor}
\begin{definition}[GCD and LCM]
    \begin{itemize}
        \item Defn. $\gcd(m, n) = \max\{k: k\mid m\land k\mid n\}$,
        \item Defn. $\text{lcm}(m, n) = \max\{k: m>0 \land m\mid k\land n\mid k\}$,
    \end{itemize}
\end{definition}
    
\end{frame}

\begin{frame}{The Euclid Algorithm}
    We assert that $\gcd(m, n) = \gcd(n , m\bmod n)$ (proof later). 

    Extended: Compute integers $m'$ and $n'$ s.t. 

    $$
    m^{\prime} m+n^{\prime} n=\operatorname{gcd}(m, n) .
    $$

    \only<1>{
    \begin{itemize}
        \item At the end of the formula, $m=0, n=\gcd(m, n)$.
        \item take $m'=0, n'=1$. 
        \item Otherwise, keep an eye on the derivation process: 
    \end{itemize}
    }

\only<2>{
    the derivation process, (q=quotient, r=remainder)

    $$
    \begin{array}{ll}
    b=r q_1+r_1 & 0 \leqslant r_1<r \\
    r=r_1 q_2+r_2 & 0 \leqslant r_2<r_1 \\
    r_1=r_2 q_3+r_3 & 0 \leqslant r_3<r_2 \\
    \cdots & \\
    r_{n-3}=r_{n-2} q_{n-1}+r_{n-1} & 0 \leqslant r_{n-1}<r_{n-2} \\
    r_{n-2}=r_{n-1} q_n+r_n & 0 \leqslant r_n<r_{n-1} \\
    r_{n-1}=r_n q_{n+1} &
    \end{array}
    $$
}

    
\end{frame}

\begin{frame}{Euclid Algo: Substitution back}
$$
    \begin{array}{ll}
    b=r q_1+r_1 & 0 \leqslant r_1<r \\
    r=r_1 q_2+r_2 & 0 \leqslant r_2<r_1 \\
    r_1=r_2 q_3+r_3 & 0 \leqslant r_3<r_2 \\
    \cdots & \\
    r_{n-3}=r_{n-2} q_{n-1}+r_{n-1} & 0 \leqslant r_{n-1}<r_{n-2} \\
    r_{n-2}=r_{n-1} q_n+r_n & 0 \leqslant r_n<r_{n-1} \\
    r_{n-1}=r_n q_{n+1} &
    \end{array}
    $$
    Hence, 
    $$
     \begin{aligned}
    d=\red{1} r_n + \red{0}\times 0 & =\red{1}r_{n-2}\red{-q_n} r_{n-1}  \\
    & \small =1 r_{n-2}-\left(r_{n-3}-q_n r_{n-2} q_{n-1}\right)  \\
    & =\red{-q_n} r_{n-3}+\red{\left(1+q_{n-1} q_n\right)} r_{n-2} \\
    & =\cdots \\
    & =\red{x} a+\red{y} b \quad(x, y \in \mathbb{Z}) .
    \end{aligned}
    $$
\end{frame}

\begin{frame}{Euclid Algo: Code}
    \begin{itemize}
        \item Use RECURSION to maintain the relation.
    \end{itemize}
    $$
    \small
\begin{aligned}
&\text { EXTENDED-EUCLID }(a, b)\\
&\begin{array}{ll}
1 & \text { if } b==0 \\
2 & \operatorname{return}(a, 1,0) \\
3 & \operatorname{else}\left(d^{\prime}, x^{\prime}, y^{\prime}\right)=\operatorname{EXTENDED}-\operatorname{EUCLID}(b, a \bmod b) \\
4 & (d, x, y)=\left(d^{\prime}, y^{\prime}, x^{\prime}-\lfloor a / b\rfloor y^{\prime}\right) \\
5 & \operatorname{return}(d, x, y)
\end{array}
\end{aligned}
$$

\only<2>{
Why $\left(d^{\prime}, y^{\prime}, x^{\prime}-\lfloor a / b\rfloor y^{\prime}\right)$? 
\begin{itemize}
    \item Condition: 
    \begin{itemize}
        \item $a x_1+b y_1=\operatorname{gcd}(a, b)$
        \item $b x_2+(a \bmod b) y_2=\operatorname{gcd}(b, a \bmod b)$
    \end{itemize}
    \item Derivation: 
    \begin{itemize}
        \item $a x_1+b y_1=b x_2+(a \bmod b) y_2$
        \item and we have that $a \bmod b=a-\left(\left\lfloor\frac{a}{b}\right\rfloor \times b\right)$
        \item So we get $a x_1+b y_1=b x_2+\left(a-\left(\left\lfloor\frac{a}{b}\right\rfloor \times b\right)\right) y_2$
        \item $a x_1+b y_1=a y_2+b x_2-\left\lfloor\frac{a}{b}\right\rfloor \times b y_2=a y_2+b\left(x_2-\left\lfloor\frac{a}{b}\right\rfloor y_2\right)$
        \item Compare the coeffs. 
    \end{itemize}
\end{itemize}
}
\only<3>{
An Example: 
\begin{center}
        \begin{tabular}{|llllrr|}
    \hline$a$ & $b$ & $\lfloor a / b\rfloor$ & $d$ & $x$ & $y$ \\
    \hline 99 & 78 & 1 & 3 & -11 & 14 \\
    78 & 21 & 3 & 3 & 3 & -11 \\
    21 & 15 & 1 & 3 & -2 & 3 \\
    15 & 6 & 2 & 3 & 1 & -2 \\
    6 & 3 & 2 & 3 & 0 & 1 \\
    3 & 0 & - & 3 & 1 & 0 \\
    \hline
    \end{tabular}
\end{center}

}
\end{frame}

\begin{frame}{Props about Divisions}

    \begin{theorem}
        $$
        (k\mid m) \wedge (k\mid n) \Leftrightarrow k\mid \operatorname{gcd}(m, n)
        $$
    \end{theorem}
\only<1>{
    \begin{itemize}
        \item follows directly from definition. 
    \end{itemize}


    \begin{theorem}[The conjuctuce of factors]
        $$
        \sum_{m \mid n} a_m=\sum_{m \mid n} a_{n \mid m} \text {, integer } n>0 \text {. }
        $$
    \end{theorem}

    \begin{itemize}
        \item Are there anything similar to this? 
    \end{itemize}

    Motivation for this (by definition): 
}
    $$
    \sum_{m | n} a_m=\sum_k \sum_{m>0} a_m[n=m k]
    $$


\only<2>{
Following above, we have 

\begin{theorem}[Interchange summation order]
    $$
    \sum_{m \mid n} \sum_{k \mid m} a_{k, m}=\sum_{k \mid n} \sum_{l \mid(n / k)} a_{k, k l}
    $$
\end{theorem}


}
    
\end{frame}

\begin{frame}{Proof for Interchange the order}
    \begin{theorem}
    $$
    \sum_{m \mid n} \sum_{k \mid m} a_{k, m}=\sum_{k \mid n} \sum_{l \mid(n / k)} a_{k, k l}
    $$
\end{theorem}

    Consider LHS: \pause
    $$
    \sum_{j, l} \sum_{{k}, m>0} a_{ k, m}[n=\blue{j} m][m={k} l]=\pause \sum_{\blue j} \sum_{k, {l}>0} a_{k, k {l}}[n=\blue{j} k l]
    $$ \pause 

    Consider RHS: \pause 
    $$
    \sum_{\red j, m} \sum_{k, l>0} a_{k, k l}[n=\red{j} k][n / k=\red m l]=\sum_{\red m} \sum_{k, l>0} a_{k, k l}[n=\red m l k]
    $$

    They are the same, standing the same meaning. 
    
\end{frame}

\begin{frame}{Interchange of Order Example}

If $k=12$: 

$$
\begin{array}{r|llllll}
m=1 & k=1 & & & \\
m=2 & k=1 & k=2 & & & \\
m=3 & k=1 & k=3 & & & \\
m=4 & k=1 & k=2 & k=4 & & \\
m=6 & k=1 & k=2 & k=3 & k=6 & \\
m=12 & k=1 & k=2 & k=3 & k=4 & k=6 & k=12
\end{array}
$$

to 
$$
\begin{array}{rlllll}
k=1 & k=2 & k=3 & k=4 & k=6 & k=12 \\
\hline
m=1 & m=2 & m=3 & m=4 & m=6 & m=12 \\
m=2 & m=4 & m=6 & m=12 & m=12 & \\
m=3 & m=6 & m=12 & & & \\
m=4 & m=12 & & & & \\
m=6 & & & & & \\
m=12 &  & & & &
\end{array}
$$
    
\end{frame}

