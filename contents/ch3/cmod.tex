\section{Mod: The binary Op}

\begin{frame}{Mod: definition}
    We may rewrite the quotient and remainder as follows: 

    If $n$ is an integer, then $$n=m\flr{n/m}+n\bmod m.$$ for $y\neq 0$. 

\only<1>{
    \begin{itemize}
        \item generalize it to negative integers
        \item $5 \bmod 3 = 5-3\flr{5/3} =2.$ 
        \item $5 \bmod -3 = 5-(-3)\flr{5/-3} =-1.$ 
        \item $-5 \bmod 3 = -5-(-3)\flr{-5/3} =1.$ 
        \item $-5 \bmod -3 = -5-(-3)\flr{-5/-3} =-2.$ 
    \end{itemize}
}

\only<2>{
    \begin{itemize}
        \item Observation: In any case the result number 
        is exactly in between 2 numbers. 
        \item Special definition: if $y=0$, then $x\bmod 0=x$. 
        \item preserves the property that $x$ and $y$ 
        always differs from $x$ by a multiple of $y$.
    \end{itemize}
}
    
\end{frame}

\begin{frame}{Another notation: Mumble}
    We have $n \bmod m = n - \flr{n/m}m$

    Alternative definition: \underline{mumble}. 
    $$
    x \operatorname{ mumble } y = y \cil{{x\over y}} -x
    $$
\end{frame}

\begin{frame}{Properties}
    \begin{itemize}
        \item Distributive: $c(x\bmod y) = (cx) \bmod (cy)$
        for $c,x,y\in \R$. 
        \item reason: $c(x\bmod y) = c(x-y\cil{x/y})=cx-cy(\flr{cx/cy}) = (cx) \bmod (cy)$. 
    \end{itemize}
\end{frame}

\begin{frame}{Example: Even partition problem(EPP)}

Problem: Partition $n$ things into $m$ groups as equally
as possible. 

\only<1>{
    An example: 
    $$
    \begin{matrix}
      1&  9&   17&  25& 33\\
      2&  10&  18&  26& 34\\
      3&  11&  19&  27& 35\\
      4&  12&  20&  28& 36\\
      5&  13&  21&  29& 37\\
      6&  14&  22&  30& \\
      7&  15&  23&  31& \\
      8&  16&  24&  32&
    \end{matrix}
    $$
   \begin{itemize}
       \item  the final row has only 5 elems, can we do better? 
   \end{itemize}
}

\only<2>{
    A evener example: 
    An example: 
    $$
    \begin{matrix}
      1&  9&   17&  24& 31\\
      2&  10&  18&  25& 32\\
      3&  11&  19&  26& 33\\
      4&  12&  20&  27& 34\\
      5&  13&  21&  28& 35\\
      6&  14&  22&  29& 36\\
      7&  15&  23&  30& 37\\
      8&  16&  &      &
    \end{matrix}
    $$
}

\only<3>{
    \begin{itemize}
        \item Division: a row by row arrange not always good. 
        \item it tells us how many lines to put 
        \begin{itemize}
            \item Some of the short one put $\cil{n/m}$ columns, others put $\flr{n/m}$ cols. 
            \item There will be exactly $n \bmod m$ cols, and exactly $m-n\bmod m = n \operatorname{ mumble }m$short ones. 
        \end{itemize}
        
    \end{itemize}
}

\only<4> {
    Procedure: 
    \begin{itemize}
        \item To distribute $n$ things into 
        $m$ groups as even as possible, 
        \item when $m>0$, put $\cil{n/m}$ things into one group
        \item then use this procedure to recursively 
        \item i.e. put put the remaining $n'=n-\cil{n-m}$ things into $m'=m-1$ groups.
    \end{itemize}
    Proof: 
    \begin{itemize}
        \item Suppose that $n=qm+r$
        \item If $r=0$, We put $\flr{n/m}=q$ things into the first, $n'= n-q, m'=m-1$. 
        \item If $r>0$, put $\flr{n/m}=q+1$ into first group, leaving $n'=n-q-1=qm'+r-1$.
    \end{itemize}
}

\only<5>{
    A closed form for the formula? 
    \begin{itemize}
        \item Effect: the quotient stays the same, but the remainder decrease by 1. 
        \item That is there are $\cil{n/m}$ things when $k\leq n \bmod m$, and $\flr{n/m}$ things o.w.
        \item So the closed form is $\cil{n-k+1/m}$. 
    \end{itemize}

    Since we are arrange $n$ elems, we have the following identity: 

    $$
    n=\flr{{n\over m}}+\flr{{n+1\over m}}+\cdots+\flr{{n+(m-1)\over m}}
    $$
    
    Replace $n$ by $mx$ we get 
    
    $$
    mx=\flr{{x}}+\flr{{x+{1\over m}}}+\cdots+\flr{{x+{m-1\over m}}}
    $$
}
\end{frame}

\begin{frame}{Example: A Weird Sum(WS)}
    Find 
    $$
    \sum_{0\leq k \leq n} \flr{\sqrt{k}}
    $$where $a$ is a perfect square. 
    
    Solution:
\only<1>{
    \begin{align*}
        &~\sum_{0\leq k\leq n} \cil{\sqrt{k}} \\  
        &=\sum_{k, m\geq 0} m[k<n][m=\cil{k}] \\
        &=\sum_{k, m\geq 0} m [k<n] [m\leq \sqrt{k}<m+1]     
    \end{align*}

    Then we calculate the total number of this.
}

\only<2>{
    \begin{align*}
        &=\sum_{k, m\geq 0} m [k<n] [m\leq \sqrt{k}<m+1]  \\
        &=\sum_{k, m\geq 0} m[m\leq k \leq (m+1)^2 \leq a^2] \\
        &= \sum_{m\geq 0} m((m+1)^2-m^2)[m+1\leq a] \\ 
        &= \sum_{m\geq 0, m\leq a} m(2m+1) \\ 
    \end{align*}

    Oh, we can use falling sums! 
}

\only<3>{
    That is 
    $$
    \sum_0^a (2m^{\underline2}+3m^{\underline1})\delta m
    $$
    Using the integration rule, we get $2/3a(a-1)(a-2)+3/2a(a-1)$. 
}

\only<4> {
    Removing the perfect square condition
    \begin{itemize}
        \item do the partition from $[0..a^2]$ and $[a^2..n]$. 
        \item this will use O notation to express its increament. 
    \end{itemize}
}
    
    
\end{frame}

\begin{frame}{Example: an Integrated Example(IE)}

Find the closed form for 
$$
\sum_{0\leq k< m} \flr{{nk+x\over m}} 
$$
for integer $m>0$, integer $n$. 

\only<1>{
    We first look at some observations
    \begin{itemize}
        \item $n=1$, yields $\sum_{0\leq k<m}\flr{(k+x)/m}$, where we found at the EPP problem. 
        \item $m=1$, this will be $\flr{x}$;
        \item $m=2$, we look at $\flr{x/2}+\flr{(x+n)/2}$. 
        \begin{itemize}
            \item $n$ even, $n/2$ integer. $\flr{x/2}+\flr{(x+n)/2}=2\flr{x/2}+{n/2}$. 
            \item $n$ odd, $(n-1)/2$ integer. $\flr{x/2}+(\flr{(x+1)/2}+(n-1)/2)=\flr{x}+{(n-1)/2}$.
        \end{itemize}
    \end{itemize}
}

\only<2>{
    Have a look at $m=3$: 
    \begin{itemize}
        \item $n \bmod 3 = 0, n/3 $ and $2n/3$ integers: $\flr{{x\over3}}+\flr{{x\over3}+{n\over 3}}+\flr{{x\over3}+{2n\over 3}}=3\flr{x/3}+n$.
        \item $n \bmod 3 = 1, n-1/3 $ and $2n-2/3$ integers: $\flr{{x\over3}}+\flr{{x+1\over3}+{n-1\over 3}}+\flr{{x+2\over3}+{2n-2\over 3}}=\flr{x}+n-1$.
        \item $n \bmod 3 = 2, n-2/3 $ and $2n-4/3$ integers: $\flr{{x\over3}}+\flr{{x+2\over3}+{n-2\over 3}}+\flr{{x+4\over3}+{2n-4\over 3}}=\flr{x}+n-1$.
    \end{itemize}
}

\only<3>{
    Look at $n=4$, 
    \begin{itemize}
        \item $n\bmod 4 = 0, 4\flr{x/4}+3n/2$;
        \item $n\bmod 4 = 1, \flr{x}+3n/2-3/2$;
        \item $n\bmod 4 = 0, \flr{x}+3n/2-3/2$;
        \item $n\bmod 4 = 0, 2\flr{x}+3n/2-1$;
    \end{itemize}
}

\only<4>{
    We make a small table for this: 
    $$
    \begin{array}{c|cccc}m &n \bmod m = 0&n \bmod m = 1&n \bmod m=2&n \bmod m==3\\
    \hline 
    1 &\lfloor x \rfloor\\
    2&2\left\lfloor\frac x2\right\rfloor+\frac n2&\lfloor x \rfloor+\frac n2-\frac12\\
    3&3\left\lfloor\frac x3\right\rfloor+n &\lfloor x \rfloor+n -1&\lfloor x \rfloor+n -1\\4&4\left\lfloor\frac x4\right\rfloor+\frac{3n }2&\lfloor x \rfloor+\frac{3n }2-\frac32&2\left\lfloor\frac x2\right\rfloor+\frac{3n }2-1&\lfloor x \rfloor+\frac{3n }2-\frac32\end{array}
    $$
    It looks that: 
    $$
    \flr{\frac{x+kn\bmod m}{m}+\frac{kn}{m}-\frac{kn\bmod m}{m}}
    $$
}

\only<5>{
    This can be extracted from
    $$
    \flr{\frac{x+kn\bmod m}{m}}+\frac{kn}{m}-\frac{kn\bmod m}{m}
    $$
}

\only<6> {
    $$
    \small
    \begin{aligned}
    & \left\lfloor\frac{x}{m}\right\rfloor&+\frac{0}{m}&-\frac{0 \bmod m}{m} \\
    + & \left\lfloor\frac{x+n \bmod m}{m}\right\rfloor&+\frac{n}{m}&-\frac{n \bmod m}{m} \\
    + & \left\lfloor\frac{x+2 n \bmod m}{m}\right\rfloor&+\frac{2 n}{m}&-\frac{2 n \bmod m}{m} \\
    &\vdots & \vdots \\
    + & \underbrace{\left\lfloor\frac{x+(m-1) n \bmod m}{m}\right\rfloor}_{a\left\lfloor\frac{x}{a}\right\rfloor}&+\underbrace{\frac{(m-1) n}{m}}_{b n}&-\underbrace{\frac{(m-1) n \bmod m}{m}}_C .
    \end{aligned}
    $$

}

\only<7>{
    Looking at the table: 
    \begin{itemize}
        \item The second column is $\frac{1}{2}\left(0+\frac{(m-1) n}{m}\right) m$
        \item The first column: See what $0 \bmod m, \quad n \bmod m, 2 n \bmod m, \cdots,(m-1) n \bmod m$ will get. 
    \end{itemize}
    
}

\only<8>{
    Look at the first row of that one, recall 
    $$
    \small
    n=\left\lfloor\frac{n}{m}\right\rfloor+\left\lfloor\frac{n+1}{m}\right\rfloor+\cdots+\left\lfloor\frac{n+(m-1)}{m}\right\rfloor
    $$

    \begin{itemize}
        \item We will encounter the remainder from $1$ to $n$ one time(we will show at Chapt. 4)
    \end{itemize}
}

\only<9> {
    So we have: 
    $$
    \small
    \begin{aligned}
    & d\left(\left\lfloor\frac{x}{m}\right\rfloor+\left\lfloor\frac{x+d}{m}\right\rfloor+\cdots+\left\lfloor\frac{x+m-d}{m}\right\rfloor\right) \\
    = & d\left(\left\lfloor\frac{x / d}{m / d}\right\rfloor+\left\lfloor\frac{x / d+1}{m / d}\right\rfloor+\cdots+\left\lfloor\frac{x / d+m / d-1}{m / d}\right\rfloor\right) \\
    = & d\left\lfloor\frac{x}{d}\right\rfloor ., \text { and hence, } a=d=\operatorname{gcd}(m, n) .
    \end{aligned}
    $$
}

\only<10>{
    The third column: $d\left(\frac{1}{2}\left(0+\frac{m-d}{m}\right) \cdot \frac{m}{d}\right)=\frac{m-d}{2}$

    \begin{itemize}
        \item $c=\frac{d-m}{2}$.
    \end{itemize}

    
}

\only<11> {
    Putting altogether: 
    $$\sum_{0 \leqslant k<m}\left\lfloor\frac{n k+x}{m}\right\rfloor=d\left\lfloor\frac{x}{d}\right\rfloor+\frac{m-1}{2} n+\frac{d-m}{2}.$$

    where $d=\operatorname{gcd}(m, n)$.

    
}

\only<12> {
    In fact, $m$ and $n$ are symmetric: 
    $$
    \small
\begin{aligned}
\sum_{0 \leqslant k<m}\left\lfloor\frac{n k+x}{m}\right\rfloor & =d\left\lfloor\frac{x}{d}\right\rfloor+\frac{m-1}{2} n+\frac{d-m}{2} \\
& =d\left\lfloor\frac{x}{d}\right\rfloor+\frac{(m-1)(n-1)}{2}+\frac{m-1}{2}+\frac{d-m}{2} \\
& =d\left\lfloor\frac{x}{d}\right\rfloor+\frac{(m-1)(n-1)}{2}+\frac{d-1}{2}
\end{aligned}
$$

saying, 

$$
\small
\sum_{0 \leqslant k<m}\left\lfloor\frac{n k+x}{m}\right\rfloor=\sum_{0 \leqslant k<n}\left\lfloor\frac{m k+x}{n}\right\rfloor, \quad \text { integers } m, n>0 \text {. }
$$
}

\end{frame}

